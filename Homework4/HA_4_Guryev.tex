\documentclass[a4paper,12pt]{article}
 \usepackage[utf8]{inputenc}
 \usepackage[left=2cm,top=1cm,right=2cm,bottom=1.5cm,nohead]{geometry}
%\usepackage{wrapfig} % Обтекание рисунков текстом
%\usepackage{floatflt}% Обтекание таблиц текстом
 \usepackage{amsmath} % Математические окружения AMS
 \usepackage{amsfonts} % Шрифты AMS
 \usepackage{amssymb} % Символы AMS
 \usepackage{listings}% to add computer code
 \usepackage{color}
 \definecolor{mygreen}{RGB}{28,172,0} % color values Red, Green, Blue
\definecolor{mylilas}{RGB}{170,55,241}
    \usepackage{graphicx} % Вставить pdf- или png-файлы
  \usepackage{euscript} % Красивый шрифт
%\usepackage{extsizes} % Возможность сделать 14-й шрифт
 \linespread{1.5} % Интерлиньяж
% \usepackage[usenames,dvipsnames,svgnames,table,rgb]{xcolor}% чтоб были гиперссылки и чтоб были цвета

\usepackage{hyperref} % Гиперссылки

%\hypersetup{
% colorlinks = true,
 %linkcolor = MidnightBlue, % ссылки на всякие разделы (их цвет)
 %urlcolor = [rgb]{0,0,1}, % чтоб задавать цыета пиксела - red green blue. не рекомендуется, если потом печатать.
 %citecolor = black
%}
%\usepackage{pdflscape}
%\oddsidemargin=10mm
%\topmargin=-15mm
\usepackage{multicol}
%\hoffset=5mm % см при печати
%\voffset=4.2mm
%\textheight = 720pt
%\textwidth=442pt

\begin{document}
\maketitle \hrulefill
\lstset{language=Matlab,%
    %basicstyle=\color{red},
    breaklines=true,%
    morekeywords={matlab2tikz},
    keywordstyle=\color{blue},%
    morekeywords=[2]{1}, keywordstyle=[2]{\color{black}},
    identifierstyle=\color{black},%
    stringstyle=\color{mylilas},
    commentstyle=\color{mygreen},%
    showstringspaces=false,%without this there will be a symbol in the places where there is a space
    numbers=left,%
    numberstyle={\tiny \color{black}},% size of the numbers
    numbersep=9pt, % this defines how far the numbers are from the text
    emph=[1]{for,end,break},emphstyle=[1]\color{red}, %some words to emphasise
    %emph=[2]{word1,word2}, emphstyle=[2]{style},
}

\begin{center}

\textbf {\Large{Empirical Methods HA 4}}\\
Konstantin Guryev\\
Pennsylvania State University\\
2018
\end{center}

\textbf{Problem \textnumero \,1 }
See the code. I use $qnwequi$ function from the CEtools. $\pi = 3.1414$ (100000 points).

\textbf{Problem \textnumero \,2 }
See the code. I use $newton\_coates$ function. $\pi = 3.1416$ (10000 points).

\textbf{Problem \textnumero \,3 }
See the code. $\pi = 3.1416$ (100000 points). The approximation is more precise with a single dimensional integral given the same number of points.


\textbf{Problem \textnumero \,4 }
See the code. $\pi = 3.1416$ (10000 points).
\vspace{\baselineskip}

\textbf{Problem \textnumero \,5 }
\vspace{\baselineskip}

$Results$ = 
\begin{pmatrix}
  0.0216& 0.0016& 0.0008 \\
  0.0104& 0.0009& 0.0000 \\
  0.0113& 0.0003& 0.0000 \\
  0.0061& 0.0002& 0.0000 
\end{pmatrix}.
\vspace{\baselineskip}

First column refers to 100 draws, second to 1000 and third to 10000 draws. The rows from 1 to 4 refer to Newton-Coates for double integral, Newton-Coates for single integral, quasi-Monte Carlo for double integral and quasi-Monte Carlo for single integral respectively.

\vspace{\baselineskip}




\newpage
\section*{Matlab Code} \lstinputlisting{qnwequi.m}

\newpage
\section*{Matlab Code} \lstinputlisting{newton_coates.m}

\newpage
\section*{Matlab Code} \lstinputlisting{HA_4_Guryev.m}

\end{document}